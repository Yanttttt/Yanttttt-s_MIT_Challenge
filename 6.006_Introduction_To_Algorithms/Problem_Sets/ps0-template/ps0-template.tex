%
% 6.006 problem set 0 solutions template
%
\documentclass[12pt,twoside]{article}

\input{macros-sp20}
\usepackage{color}
\newcommand{\theproblemsetnum}{0}

\title{6.006 Problem Set 0}

\begin{document}

\handout{Problem Set \theproblemsetnum}

\setlength{\parindent}{0pt}
\medskip\hrulefill\medskip

{\bf Name:} Connor Yan

\medskip\hrulefill

%%%%%%%%%%%%%%%%%%%%%%%%%%%%%%%%%%%%%%%%%%%%%%%%%%%%%
% See below for common and useful latex constructs. %
%%%%%%%%%%%%%%%%%%%%%%%%%%%%%%%%%%%%%%%%%%%%%%%%%%%%%

% Some useful commands:
% $f(x) = \Theta(x)$
% $T(x, y) \leq \log(x) + 2^y + \binom{2n}{n}$
% \ttt{code\_function}


% You can create unnumbered lists as follows:
% \begin{itemize}
%     \item First item in a list
%         \begin{itemize}
%             \item First item in a list
%                 \begin{itemize}
%                     \item First item in a list
%                     \item Second item in a list
%                 \end{itemize}
%             \item Second item in a list
%         \end{itemize}
%     \item Second item in a list
% \end{itemize}

% You can create numbered lists as follows:
% \begin{enumerate}
%     \item First item in a list
%     \item Second item in a list
%     \item Third item in a list
% \end{enumerate}

% You can write aligned equations as follows:
% \begin{align}
%     \begin{split}
%         (x+y)^3 &= (x+y)^2(x+y) \\
%                 &= (x^2+2xy+y^2)(x+y) \\
%                 &= (x^3+2x^2y+xy^2) + (x^2y+2xy^2+y^3) \\
%                 &= x^3+3x^2y+3xy^2+y^3
%     \end{split}
% \end{align}

% You can create grids/matrices as follows:
% \begin{align}
%     A =
%     \begin{bmatrix}
%         A_{11} & A_{21} \\
%         A_{21} & A_{22}
%     \end{bmatrix}
% \end{align}

\begin{problems}

\problem  % Problem 1

\begin{problemparts}
\problempart % Problem 1a
$\choose 5 0=1$,
$\choose 5 1=5$,
$\choose 5 2 = \frac{5!}{2!(5-2)!}= \frac{120}{12}= 10$,
$\choose 5 3 = \choose 5 2 = 10$,
$\choose 5 4 = \choose 5 1 = 5$

$\therefore A = \{1,6,9,12,13\}$

$B = \{3,6,12,15\}$,
$A \bigcap B = \{6,12\}$
\problempart % Problem 1b
$A \bigcup B=\{1,3,6,9,12,13,15\}$
$\therefore \abs{A \bigcup B}=7$
\problempart % Problem 1c
$\abs{A - B}=3$
\end{problemparts}

\problem  % Problem 2

\begin{problemparts}
\problempart % Problem 2a
$\expect{X}=\frac{1}{2^3} \times 3+2 \times \frac{1}{2^3} \times 3+3 \times \frac{1}{2^3}=\frac{12}{8}=\frac{3}{2}$
\problempart % Problem 2b
$\expect{Y}=\frac{1}{6^2}+2 \times \frac{2}{6^2}+3 \times \frac{2}{6^2} + 4 \times (\frac{2}{6^2}+\frac{1}{6^2})+5 \times \frac{2}{6^2}+ 6 \times \frac{2}{6^2} \times 2 + 8 \times \frac{2}{6^2} + 9 \times \frac{1}{6^2} + 10 \times \frac{2}{6^2} + 12 \times \frac{4}{6^2} + 15 \times \frac{2}{6^2} + 16 \times \frac{1}{6^2} + 18 \times \frac{2}{6^2} + 20 \times \frac{2}{6^2} +24 \times \frac{2}{6^2} + 25 \times \frac{1}{6^2} + 30 \times \frac{2}{6^2} + 36 \times \frac{1}{6^2} = 49/4$
\problempart % Problem 2c
$\expect{X+Y}=55/4$
\end{problemparts}

\problem  % Problem 3

\begin{problemparts}
\problempart % Problem 3a
$A=100, B=18$

$A\mod 2 = 0, B\mod 2 =0$

$\therefore$ True.
\problempart % Problem 3b
$A\mod 3 = 1, B\mod 3 =0$

$\therefore$ False.
\problempart % Problem 3c
$A\mod 4 = 0, B\mod 4 =2$

$\therefore$ False.
\end{problemparts}

\newpage
\problem  % Problem 4

\\


Prove $\sum_{i=1}^{n} i^3 = (\frac{n (n+1)}{2})^2$ where $n \geqslant 1$

As $n=1$,

$$\sum_{i=1}^{1} i^3 = 1 = \left(\frac{1 \times (1+1)}{2}\right)^2$$

Assume that the equition is true as $n=m$. Therefore,

$$\sum_{i=1}^{m} i^3 = \left(\frac{m (m+1)}{2}\right)^2$$

As $n=m+1$, 

$$
\begin {aligned}
    \sum_{i=1}^{m+1} i^3
    &= \sum_{i=1}^{m} i^3 + {(m+1)}^3\\
    &= \left(\frac{m (m+1)}{2}\right)^2+{(m+1)}^3\\
    &= \frac{m^4+6m^3+13m^2+12m+4}{4}\\
    &= \left(\frac{(m+1)(m+1+1)}{2}\right)^2
\end {aligned}
$$

Hence, for any positive integer $n$, 
$\sum_{i=1}^{n} i^3 = {\left(\frac{n (n+1)}{2}\right)}^2$ is true.

\newpage
\problem  % Problem 5

Prove:

Let $G_1=(V_1,E_1)$, where $\abs{V_1=1}$ and $\abs{E_1=0}$.
Since there are only a vertex in the graph, it is apparently acyclic.

Assume that a graph $G_k=(V_k,E_k)$ that has $m$ vertices and $m-1$ edges is acyclic. 
If we now add a edge and a vertex, constructing a graph with $k+1$ vertices and $k$ edges, 
since the new vertex cannot form a cycle with those original vertices 
the new graph is still acyclic.

Hence, any connected undirected graph $G=(V,E)$, for wich$\abs{V}-1=\abs{E}$ is acyclic.



\textcolor{red}{
 Solution: 
 Induct on the number of vertices k. Base Case:
 a graph containing one vertex and zero edges is trivially acyclic. 
 Now assume for induction the claim is true 
 for any connected graph having exactly k vertices and k − 1 edges,
 and consider any connected graph G containing exactly k + 1 vertices and k edges. G is connected 
 so every vertex connects to at least one edge. 
 Since each of the k edges connects to two vertices, the average degree of vertices in G is 2k/(k + 1) < 2, 
 so there exists at least one vertex v with degree 1, connected to exactly one vertex u. 
 Removing v and the edge connecting v to u yields a graph G0 on k vertices and k − 1 edges that is also connected. 
 Vertex v cannot be in any cycle of G since a vertex in a cycle has degree at least 2, 
 so G contains a cycle only if G0 contains a cycle. 
 By the inductive hypothesis G0 is acyclic, so G is also.
}



\vfill
\problem  % Problem 6
Submit your implementation to {\small\url{alg.mit.edu}}.

\begin{lstlisting}
def count_long_subarray(A):
    '''
    Input:  A     | Python Tuple of positive integers
    Output: count | number of longest increasing subarrays of A
    '''
    count = 0
    ##################
    # YOUR CODE HERE #
    ##################
    return count
\end{lstlisting}

\end{problems}

\end{document}
