%
% 6.006 problem set 1 solutions template
%
\documentclass[12pt,twoside]{article}

\input{macros-sp20}
\newcommand{\theproblemsetnum}{1}

\title{6.006 Problem Set 1}

\begin{document}

\handout{Problem Set \theproblemsetnum}

\setlength{\parindent}{0pt}
\medskip\hrulefill\medskip

{\bf Name:} Your Name

\medskip

{\bf Collaborators:} None

\medskip\hrulefill

%%%%%%%%%%%%%%%%%%%%%%%%%%%%%%%%%%%%%%%%%%%%%%%%%%%%%
% See below for common and useful latex constructs. %
%%%%%%%%%%%%%%%%%%%%%%%%%%%%%%%%%%%%%%%%%%%%%%%%%%%%%

% Some useful commands:
%$f(x) = \Theta(x)$
%$T(x, y) \leq \log(x) + 2^y + \binom{2n}{n}$
% {\tt code\_function}


% You can create unnumbered lists as follows:
%\begin{itemize}
%    \item First item in a list
%        \begin{itemize}
%            \item First item in a list
%                \begin{itemize}
%                    \item First item in a list
%                    \item Second item in a list
%                \end{itemize}
%            \item Second item in a list
%        \end{itemize}
%    \item Second item in a list
%\end{itemize}

% You can create numbered lists as follows:
%\begin{enumerate}
%    \item First item in a list
%    \item Second item in a list
%    \item Third item in a list
%\end{enumerate}

% You can write aligned equations as follows:
%\begin{align}
%    \begin{split}
%        (x+y)^3 &= (x+y)^2(x+y) \\
%                &= (x^2+2xy+y^2)(x+y) \\
%                &= (x^3+2x^2y+xy^2) + (x^2y+2xy^2+y^3) \\
%                &= x^3+3x^2y+3xy^2+y^3
%    \end{split}
%\end{align}

% You can create grids/matrices as follows:
%\begin{align}
%    A =
%    \begin{bmatrix}
%        A_{11} & A_{21} \\
%        A_{21} & A_{22}
%    \end{bmatrix}
%\end{align}

% You can include images and PDFs as follows:
% \includegraphics[width=0.5\textwidth]{img.jpg}

\begin{problems}

\problem  % Problem 1

\begin{problemparts}
\problempart % Problem 1a
$f_5, f_3, f_4, f_1, f_2$
\problempart % Problem 1b
$\{f_1,f_2\},f_5,\{f_3,f_4\}$
\problempart % Problem 1c

$f_2=\frac{n!}{n! 6!}=1/(6!)$

$f_4=\frac{n!}{(n/6)! (5n/6)!}$

$f_2,f_5,\{f_3,f_4\},f_1$ 

\problempart % Problem 1d

$f_5,f_2,f_4,\{f_1,f_3\}$
\end{problemparts}

\newpage
\problem  % Problem 2

\begin{problemparts}
\problempart % Problem 2a
For $j$ equals $i$ to $i+k/2$, do \verb|insert_at(j, delete_at(2*i+k-j-1))|.

\problempart % Problem 2b
For $j$ equals $0$ to $k-1$, do \verb|insert_at(j, delete_at(i+j))|.

\end{problemparts}

\newpage
\problem  % Problem 3

% \verb|build(X)|: build a static array, for each position we store a page of note. Run in $O(|X|)$ time.



%\verb|read_page(i)|:

\newpage
\problem  % Problem 4



``$\rightarrow$'' means point to

\begin{problemparts}
\problempart % Problem 4a


\verb|insert_first(x)|: $x.next\rightarrow L.head, L.head.prev\rightarrow x, x.prev=null, L.head\rightarrow x$

\verb|insert_last(x)|: $x.prev\rightarrow L.tail, x.next=null, L.tail.next\rightarrow x, L.tail\rightarrow x$

\verb|delete_first()|: $L.head.next.prev=null, L.head\rightarrow L.head.next$

\verb|delete_last()|: $L.tail.prev.next=null, L.tail\rightarrow L.tail.prev$


\problempart % Problem 4b

$L'.head \rightarrow x_1, L'.tail \rightarrow x_2$

$x_1.prev.next \rightarrow x_2.next, x_2.next.prev\rightarrow x_1.prev$

$x_1.prev=null,x_2.next=null$

\problempart % Problem 4c


$L_2.tail.next \rightarrow x.next$
$x.next.prev \rightarrow L_2.tail$
$L_2.head.prev \rightarrow x$
$x.next \rightarrow L_2.head$

\problempart Submit your implementation to {\small\url{alg.mit.edu}}.
\end{problemparts}

\end{problems}

\end{document}
